\newpage
\begin{minipage}[b]{0.47\textwidth}
   \normalsize{
   что она обязана повторить вызов.\\
   Кроме того, жюри имеет право часть\\
   очков не распределять между коман-\\
   дами вообще.

   \qquadРасписание ролей команд надо\\
    составлять так, чтобы каждая коман-\\
    да могла вызвать каждую другую.\\
    Если бой ведется по 6 (или 12) зада-\\
    чам, то в 6 турах как раз получаются\\
    все перестановки 3 команд (3!=6):\\
    АВС, ВСА, САВ, АСВ, СВА, ВАС.\\

    \textbf{Задачи математического боя в Че-\\
    лябинске на Всесоюзной математи-\\
    ческой олимпиаде *)\\}

    \qquad \textbf{1.}\textit{Дана последовательность} \{$a_n$\}\\
     \textit{и функция f такая, что .f (n+1) -\\
     - f (n) $\ge$ n + 1. Известно, что $a_n$ $\le$\\
     $\le$ $a_{n+1}$} + \textit{$a_{f(n)}$. Докажите, что можно\\
     указать такие члены \indent  $a_i$ , \ldots\\
     \dots , $a_{i_k}$ , что $a_{i_1}$}+\textit{$a_{i_2}$} +   \ldots + \textit{$a_{i_k}$}> 100.

    \qquad \textbf{2.}\textit{В сыре, имеющем форму куба.\\
    $\text{n}\times \text{n}\times \text{n}$, вырезана сферическая дыр-\\
    ка диаметра 1. Найти минимальное\\
    число плоских разрезов, позволяющих\\
    наверняка ее обнаружить.}

    \qquad \textbf{3.}\textit{Каждая страна на плоскости\\
    состоит из одного или двух кусков.}\\}
    \noindent\rule{2cm}{0.4pt}\\
    \scriptsize*)\textit{Эту часть статьи подготовил к печати\\
     председатель жюри математического боя\\
     Л.Г. Лиманов.}\\
 \end{minipage}
\begin{minipage}[b]{0.47\textwidth}
    \normalsize\textit{Докажите, что карту можно пра-\\
    вильно раскрасить 12 цветами.}

    \qquad \textbf{4.}\textit{В треугольнике ABC построены\\
    внутренним образом равнобедренные\\
    треугольники ABC', BCA', ACB'.\\
    Доказать, что прямые C$C_1$ , B$B_1$ и\\
    A$A_1$ , перпендикулярные A'B' , A'C' и\\
    B'C' соответственно, пересекаются\\
    в одной точке.}

    \qquad \textbf{5.}\textit{Для всякого n можно указать\\
     такое m, что из m человек можно\\
     выбрать n попарно знакомых или n\\
     попарно незнакомых.}

    \qquad \textbf{6.} \textit{Даны числа $a_0$, $a_1$, $a_2$, \ldots\\
     \ldots, $a_n$, причем $a_0$ = $a_n$ = 0, $a_i$ > 0 при\\
     i $\not=$0, n и $\frac{a_{s-1} + a_{s+1}}{2}$ $\ge$ $a_s$ $\cos$ $\frac{\pi}{k}$.\\
     Доказать, что n $\ge$ k.}

     \qquad \textbf{7.}\textit{Дана функция f на отрезкe}[\textit{ab}],\\
     \textit{причем}

     \qquad\textit{ f + f >0,     f(a) = f(b)= 0,\\
     \indent \indent \indent f(x)> 0 на (ab).\\
     Доказать, что b - a $\ge$ $\pi$} *).

    \qquad \textbf{8.} \textit{Пусть a и n - наутральные\\
     числа, большие 1. Доказать, что\\
    \indent \indent \indent $a^n$ - a $\not =$ $\displaystyle\sum_d\frac{a^n - 1}{a^d - 1}$ ,\\
     где суммирование ведется по не-\\
     которым делителям d числа n.}\\
     \noindent\rule{2cm}{0.4pt}\\
     \scriptsize*)\textit{f'' - вторая производная функции f.\\
     Считается, что она существует во всех точках\\
     отрезка} [\textit{a}, \textit{b}].\\
\end{minipage}
 \begin{center}
      \par\noindent\rule{\textwidth}{0.4pt}
      \small\textbf{В $\ll$Кванте$\gg$ № 6 следующие опечатки:}
   \end{center}
   \begin{center}
     \begin{tabular}{ |c|c|c|c|c| } 
       стр. &  &  & напечатано & должно быть\\
       \hline
       75 &правая колонка & 14 строка снизу & 5a $\pm\sqrt{a^2 + 4a}$ & 5a - $\sqrt{a^2 + 4a}$\\
       77 & левая колонка & 8 строка снизу & 20$\sqrt{21}$, 20$\sqrt{21}$ & 10$\sqrt{21}$, 10$\sqrt{21}$ \\
       77 & правая колонка & 7 строка сверху & \textit{S} & \textit{2S}\\
     \end{tabular}
 \end{center}
